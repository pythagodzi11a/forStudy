%% 前置部分--frontmatter.tex
%% 诚信声明
\makedeclare%[figure/declare-master-doctor.png]

%% 摘要
\begin{cabstract}
	摘要和关键词一起写在这里。

	摘要是学位论文的内容不加注释和评论的简短陈述,置于学位论文数据集后。
	摘要应具有独立性和自含性,即不阅读论文的全文,
	就能获得必要的信息。摘要中有数据、结论,是一篇完整的短文,
	可以独立使用和引用。摘要的内容应包含与论文正文等同量的主
	要信息,供读者确定有无必要阅读全文,也可供二次文献(文摘
	等)采用。摘要一般应说明研究工作的目的、实验方法、结果和
	最终结论等,重点突出具有创新性的成果和新见解。
	
	硕士论文摘要 500 字左右,博士论文摘要 1500 字左右。
	英文摘要应与中文摘要内容一致。除无法变通的办法可用以外,
	摘要中不用图、表、化学结构式、非公知公用的符号和术语。
	
	接下来的一大段废段话的作用是将中文摘要写到 1500 个字。
	
	\zhlipsum[1-3]
	
	本项目的创新点有:
	\begin{enumerate}
		\item 开发了第一份适用于北京化工大学毕业论文的 \LaTeX\ 模板;
		\item 以自身为示例展示此模板的使用方法;
		\item 这是编号列表环境的第三项。
	\end{enumerate}

	(以上共约 1600 字)
\end{cabstract}

\begin{eabstract}
	Here is the Abstract and the Keywords.

	The followings are nonsense.

	\lipsum[1-7]

	Innovations in the research:
	\begin{itemize}
		\item Developing the first \LaTeX{} writting template for BUCT undergraduate thesis;
		\item Using the PDF itself as an example to show how to use the template;
		\item This is the third item of an unnumbered list.
	\end{itemize}

	(Around 5000 letters above)
\end{eabstract}