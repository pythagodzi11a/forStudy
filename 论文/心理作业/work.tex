\documentclass{ctexart}

\begin{document}

\title{大学生身心健康课试题}
\author{周昭宇}
\date{\today}

\pagestyle{empty}

\begin{titlepage}
    \maketitle
\end{titlepage}

\tableofcontents

\newpage

\section{我校心理咨询中心的地址和预约电话(东区、北区)是什么?}
北区心理咨询室地址:后勤服务楼401室。\par
北区预约电话:80191025\par
东区心理咨询室地址:科技大厦315室\par
东区预约电话:64413344\par

\section{请描述你熟悉的文学/影视作品人物或你自己的一段成长经历或一个重大事件,并运用课上所学内容用心理学的视角对其进行简要分析}
我想讨论关于《简·爱》中的主人公简·爱的家庭对她的影响。简·爱在童年时期失去了父母,被寄养在舅母的家里,受到冷漠和虐待。这种家庭环境对她的依恋模式产生了深远影响,可能使她形成了回避型或焦虑型依恋。简·爱早期的情感忽视导致她在成年后形成了强烈的独立性和自我保护机制,她学会了依靠自己而非依赖他人,这体现了她的“回避型依恋”倾向。她渴望被爱,但同时又有着自我防卫的心理,避免过度依附。这种早期缺失的亲情塑造了简·爱的坚韧个性,并促使她将自尊和自爱作为重要的心理防线。她追求爱情时,总是把自我价值和独立放在第一位,这使她在面对罗切斯特时,既有深情的依恋,也坚持自己的独立和尊严。\par
在简·爱与罗切斯特的爱情中,初次相遇时她体验到了典型的爱情生理反应:心跳加速、情感澎湃等。心理学研究表明,这些反应主要由肾上腺素等神经递质引起,是身体对情感刺激的自然反应。这种反应通常发生在爱情的初期,可能导致一种错觉,认为自己与对方有着强烈的“命中注定”的联系。

然而,这种生理反应并不一定代表爱情的深度,更多的是由生物化学反应推动的短期情感刺激。因此,简·爱在与罗切斯特的关系中,虽然有强烈的情感波动,但她逐渐学会理性对待这段感情。她意识到,真正的爱情不仅仅是生理反应和激情,还需要理智和责任。

\section{围绕本课程所学主体,选取让你感触最深的部分,谈谈感受}
让我感受最深的部分是爱情专题。当然这不仅仅是因为他是最后一课,更是因为爱情专题能够给大学生们提供切实的帮助。之前我对爱情的理解,更多停留在一种浪漫和理想化的层面——认为爱情是一种纯粹的情感体验,伴随激情和心跳加速,似乎是命中注定的美好。然而,在这门心理健康课上,老师从心理学角度解析了恋爱中的情感和生理反应,让我对爱情有了更为理性、全面的认知。\par
而从心理学的角度,爱情不仅仅是情感的体验,更是依赖于依附理论、吸引力法则等心理机制。课程让我深入探讨了爱情的多维性,帮助我更好地理解自己和他人之间的情感关系。\par
总的来说,爱情专题让我对爱情有了全新的认知。爱情不仅仅是浪漫和激情,它涉及生理反应、心理需求和情感发展。在今后的恋爱中,我希望能更理性地看待感情,同时理解其中的生理和心理机制,避免被激情冲昏头脑,保持理智和清醒。尽管爱情充满迷茫,但通过心理学的知识,我相信可以更健康地面对它。

\section{你目前面临哪些压力?根据课堂所讲授的内容,以及自己以往应对压力的经验,谈一谈如何能够更好地应对当前的压力。}
目前我面临的压力主要来自学业和时间管理。课程任务较多,截止日期紧迫,这让我时常感到焦虑和紧张,尤其是在复习和写作业时,总是觉得时间不够用,自己跟不上进度。\par
在心理学课程中,我学到的一个重要点是识别压力源。比如说,当前压力的主要来源是即将到来的考试和作业的截止日期。因此,我需要合理安排时间,分解任务,每天设定明确的小目标,一步一步完成。这种方法可以减少对整体任务的焦虑感,逐步推进让我有更强的控制感,也能缓解压力。\par
此外,我还意识到情绪管理非常重要。面对压力时,我常会感到不安,尤其是在面对复杂的任务时。为此,我尝试通过深呼吸、冥想或者简单的运动来放松身心。这样能让我暂时抽离焦虑情绪,冷静下来重新审视自己的状态,确保不会被负面情绪主导。\par
我还学到了“社会支持”的重要性。和朋友、同学讨论问题或寻求帮助,不仅能解决实际问题,还能让我感到自己并不孤单。心理学课上提到,倾诉和分享压力能显著减轻负担,这让我感到心里更有底。\par
另外,我也开始注重长期的压力管理,保持良好的作息和健康的饮食,避免长期处于高压状态,防止压力对身体和心理的负面影响。总之,通过这门课程,我学到了如何更理性地面对压力,调整自己的心态,采取有效的方法进行管理,使我能够更加冷静地应对挑战。\par

\end{document}